\chapter{Aufgabenstellung}

Ziel dieser Laborübung ist die praktische Umsetzung einer \textbf{Public Key Infrastructure (PKI)} mithilfe der Software \textbf{XCA} (\emph{X Certificate and Key Management}).
Dabei sollen die grundlegenden Schritte zur \textbf{Erstellung, Verwaltung und Überprüfung digitaler Zertifikate} nachvollzogen und in einer realen Umgebung getestet werden.

\bigskip

Im Rahmen der Übung wird zunächst eine einfache PKI-Struktur aufgebaut, bestehend aus einem Root-, Intermediate- und Serverzertifikat.
Anschließend wird die Funktionsweise der \emph{Certificate Chain}, der Widerrufsmechanismen und der Zertifikatsprüfung demonstriert.

\bigskip
In einer virtuellen Linux-Umgebung wird ein \textbf{Webserver} (z.\,B. \emph{Nginx}) installiert und mit dem erzeugten Serverzertifikat konfiguriert.
Auf dem Client-Rechner werden anschließend die notwendigen Zertifikate importiert, um die \textbf{Client-Server-Kommunikation über TLS} zu testen.

\section*{Erweiterte Aufgabe im Klassenverbund}
Im weiteren Verlauf der Übung wird das Verhalten verschiedener Webbrowser bei widerrufenen Zertifikaten untersucht:
\begin{itemize}
    \item Microsoft Edge
    \item Mozilla Firefox
    \item Google Chrome
    \item \textbf{Apple Safari}
\end{itemize}

Das Ziel besteht darin, den Zertifikatsfehler zu erkennen und durch das Erstellen sowie Importieren neuer Zertifikate zu beheben.

\bigskip
Die Vorgehensweise, die Ergebnisse und die gewonnenen Erkenntnisse im Laborbericht dokumentiert.
