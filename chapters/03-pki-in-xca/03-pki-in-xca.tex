\chapter{Konfiguration einer PKI in XCA}

\section{Installation von XCA}

Für die Konfiguration der Public Key Infrastructure (PKI) wird eine virtuelle Maschine mit \textbf{Kali Linux} verwendet.
In der Regel ist \textbf{XCA} (\emph{X Certificate and Key Management}) bereits in der Standardinstallation von Kali Linux enthalten.
Sollte dies nicht der Fall sein, kann die Installation über die Paketverwaltung erfolgen:

\begin{tcolorbox}[colback=black!3!white, colframe=black!60!white]
    \begin{verbatim}
sudo apt update
sudo apt install xca
\end{verbatim}
\end{tcolorbox}


Nach der Installation kann XCA über das Anwendungsmenü oder den Terminalbefehl \texttt{xca} gestartet werden.

\bigskip

\section{Erstellen der Zertifikatsdatenbank}

Um Zertifikate verwalten und signieren zu können, muss zunächst eine neue \textbf{Zertifikatsdatenbank} erstellt werden.
Dies erfolgt in XCA über das Menü:

\begin{quote}
    \texttt{File → New Database}
\end{quote}

Im folgenden Dialog wird der Datenbank ein aussagekräftiger Name gegeben sowie ein Speicherort festgelegt.
Das Passwortfeld bleibt leer.

\begin{figure}[H]
    \centering
    \includegraphics[width=0.8\linewidth]{01-database.png}
    \caption{Certificate Database created}
    \label{fig:screenshot01}
\end{figure}

Sobald die Datenbank angelegt wurde, kann mit der Erstellung der obersten Instanz begonnen werden: dem \textbf{Root-Zertifikat}.

\bigskip

\section{Erstellung der benötigten Zertifikate}

Im nächsten Schritt werden die für die PKI-Struktur erforderlichen Zertifikate erzeugt.
Dazu gehören:

\begin{enumerate}[label=\alph*)]
    \item ein \textbf{Root-Zertifikat} (selbstsigniert),
    \item ein \textbf{Intermediate-Zertifikat}, das vom Root-Zertifikat signiert wird,
    \item ein \textbf{Server-Zertifikat (TLS)}, das vom Intermediate-Zertifikat signiert wird.
\end{enumerate}

Alle Zertifikate werden mit modernen \textbf{RSA}-Schlüsseln erstellt (nicht mit einem ED25519-Schlüssel)

\bigskip

Die detaillierte Vorgehensweise zur Erstellung der einzelnen Zertifikate wird in den folgenden Abschnitten beschrieben:

\newpage
\subsection{Root CA Certificate (Certificate Authority)}

Um ein Root-CA-Zertifikat zu erstellen, wechselt man zunächst in den \emph{Certificates}-Tab und klickt auf:
\begin{quote}
    \texttt{Certificates → New Certificate}
\end{quote}

\begin{figure}[H]
    \centering
    \includegraphics[width=0.8\linewidth]{02-Certificates-Tab.png}
    \caption{Creating a Root CA Certificate}
    \label{fig:screenshot02}
\end{figure}

Daraufhin öffnet sich das Fenster zur Konfiguration des neuen Zertifikats.
Im \emph{Source}-Tab muss unter \emph{Signing} die Option \emph{"Create a self-signed certificate"} aktiviert werden.

\begin{figure}[H]
    \centering
    \includegraphics[width=0.8\linewidth]{03-create-ca-certificate.png}
    \caption{Creating a Root CA Certificate – \emph{Source}-Tab}
    \label{fig:screenshot03}
\end{figure}

Anschließend wechselt man in den \emph{Subject}-Tab, um die Informationen für das Zertifikat einzugeben.
In diesem Schritt wird außerdem der zugehörige Private Key erstellt. Im unteren Bereich des \emph{Subject}-Tabs befindet sich die Sektion \emph{Private Key}. Hier wird ein neuer Schlüssel mit den unten dargestellten Einstellungen generiert.

\begin{figure}[h!]
    \centering
    \begin{subfigure}[b]{0.4\linewidth}
        \includegraphics[width=\linewidth]{04-ca-subject-tab.png}
        \caption{Information of the CA Certificate}
    \end{subfigure}
    \begin{subfigure}[b]{0.4\linewidth}
        \includegraphics[width=\linewidth]{05-CA-Private-Key.png}
        \caption{Private Key for CA Certificate}
    \end{subfigure}
    \caption{Creating a Root CA Certificate – \emph{Subject}-Tab}
    \label{fig:screenshot45}
\end{figure}

Im nächsten Schritt wird im \emph{Extensions}-Tab der Zertifikatstyp festgelegt. Zusätzlich muss der \emph{x509v3 CRL Distribution Point} eingetragen werden.

\begin{figure}[H]
    \centering
    \includegraphics[width=0.8\linewidth]{06-CA-Extension-Tab.png}
    \caption{Creating a Root CA Certificate – \emph{Extension}-Tab}
    \label{fig:screenshot06}
\end{figure}

Nach erfolgreicher Erstellung erscheint das neue Root-Zertifikat in der Datenbank unter \emph{Certificates}.

\begin{figure}[H]
    \centering
    \includegraphics[width=0.8\linewidth]{07-CA-in-DB.png}
    \caption{Root-CA Certificate in Database}
    \label{fig:screenshot07}
\end{figure}

\newpage
\subsection{Intermediate Certificate}

Das \textbf{Intermediate-Zertifikat} wird analog zum Root-Zertifikat erstellt. Der entscheidende Unterschied besteht darin, dass es \emph{nicht selbstsigniert} ist, sondern vom Root-Zertifikat signiert wird.

Im Dialogfeld für die Signatur wird unter \emph{Signing} die Option \emph{Use this Certificate for signing} ausgewählt und anschließend das zuvor erstellte Root-Zertifikat als Aussteller angegeben.
Als Vorlage kann unter \emph{Template for the new certificate} das \emph{CA}-Template verwendet werden, da das Intermediate-Zertifikat ebenfalls als Zertifizierungsstelle fungiert.

\begin{figure}[H]
    \centering
    \includegraphics[width=0.8\linewidth]{08-intermediate-source-tab.png}
    \caption{Intermediate Certificate – \emph{Source}-Tab}
    \label{fig:screenshot08}
\end{figure}

Im Tab \emph{Subject} werden die Identitätsinformationen für das Intermediate-Zertifikat eingetragen.
Hier wird ein neuer privater Schlüssel erzeugt — wichtig ist, dass dieser als \textbf{RSA-Schlüssel} erstellt wird (kein ED25519).

\begin{figure}[H]
    \centering
    \begin{subfigure}[b]{0.45\linewidth}
        \includegraphics[width=\linewidth]{09-Intermediate-Source-Tab.png}
        \caption{Intermediate Certificate – Subject Information}
    \end{subfigure}
    \hfill
    \begin{subfigure}[b]{0.45\linewidth}
        \includegraphics[width=\linewidth]{10-Intermediate-Private-Key.png}
        \caption{Generated RSA Private Key}
    \end{subfigure}
    \caption{Configuration of the Intermediate Certificate – \emph{Subject}-Tab}
    \label{fig:screenshot910}
\end{figure}

Anschließend wird – wie beim Root-Zertifikat – in den Tab \emph{Extensions} gewechselt, um den Zertifikatstyp festzulegen.
Da das Intermediate-Zertifikat ebenfalls eine Zertifizierungsstelle darstellt, wird hier die Option \emph{Certification Authority} ausgewählt.

\begin{figure}[H]
    \centering
    \includegraphics[width=0.8\linewidth]{11-Intermediate-Extension.png}
    \caption{Intermediate Certificate – \emph{Extensions}-Tab}
    \label{fig:screenshot11}
\end{figure}


\newpage
\subsection{TLS Server Certificate}

Im letzten Schritt wird das \textbf{TLS-Server-Zertifikat} erstellt. Dieses Zertifikat wird später auf dem Webserver eingebunden und dient der verschlüsselten Kommunikation mit den Clients.

Die Erstellung erfolgt analog zu den vorherigen Zertifikaten.
Als Signaturzertifikat wird diesmal jedoch das zuvor erzeugte \textbf{Intermediate-Zertifikat} verwendet.
Unter \emph{Template for the new certificate} kann das vordefinierte \emph{TLS Server}-Template ausgewählt werden.

\begin{figure}[H]
    \centering
    \includegraphics[width=0.8\linewidth]{12-TLS-Zertifikat-Source-Tab.png}
    \caption{TLS Certificate – \emph{Source}-Tab}
    \label{fig:screenshot12}
\end{figure}

Im Tab \emph{Subject} werden die Identitätsdaten des Server-Zertifikats eingetragen.
Als \emph{Common Name (CN)} muss hier die \textbf{IP-Adresse der virtuellen Maschine} angegeben werden, auf der der Webserver (\emph{nginx}) betrieben wird.
Auch dieses Zertifikat erhält einen \textbf{RSA-Schlüssel} – es soll kein ED25519-Schlüssel verwendet werden.

\begin{figure}[H]
    \centering
    \includegraphics[width=0.8\linewidth]{13-TLS-Zertifikat-Subject-Tab.png}
    \caption{TLS Certificate – \emph{Subject}-Tab}
    \label{fig:screenshot13}
\end{figure}

Anschließend wird in den Tab \emph{Extensions} gewechselt, um die Zertifikatseigenschaften festzulegen.
Unter \emph{X.509v3 Basic Constraints} wird der Typ \textbf{End Entity} ausgewählt, da dieses Zertifikat keine weiteren Zertifikate signiert.

Zusätzlich werden im unteren Bereich der Maske der \emph{Subject Alternative Name (SAN)} sowie der \emph{CRL Distribution Point} eingetragen.
Wenn im SAN eine IP-Adresse verwendet wird – was hier der Fall ist – muss diese korrekt als IP angegeben werden, also in der Form:

\begin{quote}
    \texttt{IP:192.168.40.144}
\end{quote}

und nicht als URI, also \texttt{URI:192.168.40.144}.

\begin{figure}[H]
    \centering
    \begin{subfigure}[b]{0.45\linewidth}
        \includegraphics[width=\linewidth]{14-TLS-Zertifikat-Extension-Tab.png}
        \caption{TLS Certificate – \emph{Extensions}-Tab}
    \end{subfigure}
    \hfill
    \begin{subfigure}[b]{0.45\linewidth}
        \includegraphics[width=\linewidth]{15-TLS-Zertifikat-SAT-IP.png}
        \caption{Subject Alternative Name (IP-Adresse)}
    \end{subfigure}
    \caption{TLS Certificate – Einstellungen im \emph{Extensions}-Tab}
    \label{fig:screenshot1415}
\end{figure}


\newpage
Nachdem alle drei Zertifikate erstellt wurden, sollte die Zertifikatsdatenbank in XCA wie in Abbildung \ref{fig:screenshot16} aussehen.

\begin{figure}[H]
    \centering
    \includegraphics[width=0.8\linewidth]{16-All-Certificates.png}
    \caption{All Certificates in Certificate-Database}
    \label{fig:screenshot16}
\end{figure}