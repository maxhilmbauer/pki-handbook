\subsection{Root CA Certificate (Certificate Authority)}

Um ein Root-CA-Zertifikat zu erstellen, wechselt man zunächst in den \emph{Certificates}-Tab und klickt auf:
\begin{quote}
    \texttt{Certificates → New Certificate}
\end{quote}

\begin{figure}[H]
    \centering
    \includegraphics[width=0.8\linewidth]{02-Certificates-Tab.png}
    \caption{Creating a Root CA Certificate}
    \label{fig:screenshot02}
\end{figure}

Daraufhin öffnet sich das Fenster zur Konfiguration des neuen Zertifikats.
Im \emph{Source}-Tab muss unter \emph{Signing} die Option \emph{"Create a self-signed certificate"} aktiviert werden.

\begin{figure}[H]
    \centering
    \includegraphics[width=0.8\linewidth]{03-create-ca-certificate.png}
    \caption{Creating a Root CA Certificate – \emph{Source}-Tab}
    \label{fig:screenshot03}
\end{figure}

Anschließend wechselt man in den \emph{Subject}-Tab, um die Informationen für das Zertifikat einzugeben.
In diesem Schritt wird außerdem der zugehörige Private Key erstellt. Im unteren Bereich des \emph{Subject}-Tabs befindet sich die Sektion \emph{Private Key}. Hier wird ein neuer Schlüssel mit den unten dargestellten Einstellungen generiert.

\begin{figure}[h!]
    \centering
    \begin{subfigure}[b]{0.4\linewidth}
        \includegraphics[width=\linewidth]{04-ca-subject-tab.png}
        \caption{Information of the CA Certificate}
    \end{subfigure}
    \begin{subfigure}[b]{0.4\linewidth}
        \includegraphics[width=\linewidth]{05-CA-Private-Key.png}
        \caption{Private Key for CA Certificate}
    \end{subfigure}
    \caption{Creating a Root CA Certificate – \emph{Subject}-Tab}
    \label{fig:screenshot45}
\end{figure}

Im nächsten Schritt wird im \emph{Extensions}-Tab der Zertifikatstyp festgelegt. Zusätzlich muss der \emph{x509v3 CRL Distribution Point} eingetragen werden.

\begin{figure}[H]
    \centering
    \includegraphics[width=0.8\linewidth]{06-CA-Extension-Tab.png}
    \caption{Creating a Root CA Certificate – \emph{Extension}-Tab}
    \label{fig:screenshot06}
\end{figure}

Nach erfolgreicher Erstellung erscheint das neue Root-Zertifikat in der Datenbank unter \emph{Certificates}.

\begin{figure}[H]
    \centering
    \includegraphics[width=0.8\linewidth]{07-CA-in-DB.png}
    \caption{Root-CA Certificate in Database}
    \label{fig:screenshot07}
\end{figure}