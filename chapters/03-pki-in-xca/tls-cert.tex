\subsection{TLS Server Certificate}

Im letzten Schritt wird das \textbf{TLS-Server-Zertifikat} erstellt. Dieses Zertifikat wird später auf dem Webserver eingebunden und dient der verschlüsselten Kommunikation mit den Clients.

Die Erstellung erfolgt analog zu den vorherigen Zertifikaten.
Als Signaturzertifikat wird diesmal jedoch das zuvor erzeugte \textbf{Intermediate-Zertifikat} verwendet.
Unter \emph{Template for the new certificate} kann das vordefinierte \emph{TLS Server}-Template ausgewählt werden.

\begin{figure}[H]
    \centering
    \includegraphics[width=0.8\linewidth]{12-TLS-Zertifikat-Source-Tab.png}
    \caption{TLS Certificate – \emph{Source}-Tab}
    \label{fig:screenshot12}
\end{figure}

Im Tab \emph{Subject} werden die Identitätsdaten des Server-Zertifikats eingetragen.
Als \emph{Common Name (CN)} muss hier die \textbf{IP-Adresse der virtuellen Maschine} angegeben werden, auf der der Webserver (\emph{nginx}) betrieben wird.
Auch dieses Zertifikat erhält einen \textbf{RSA-Schlüssel} – es soll kein ED25519-Schlüssel verwendet werden.

\begin{figure}[H]
    \centering
    \includegraphics[width=0.8\linewidth]{13-TLS-Zertifikat-Subject-Tab.png}
    \caption{TLS Certificate – \emph{Subject}-Tab}
    \label{fig:screenshot13}
\end{figure}

Anschließend wird in den Tab \emph{Extensions} gewechselt, um die Zertifikatseigenschaften festzulegen.
Unter \emph{X.509v3 Basic Constraints} wird der Typ \textbf{End Entity} ausgewählt, da dieses Zertifikat keine weiteren Zertifikate signiert.

Zusätzlich werden im unteren Bereich der Maske der \emph{Subject Alternative Name (SAN)} sowie der \emph{CRL Distribution Point} eingetragen.
Wenn im SAN eine IP-Adresse verwendet wird – was hier der Fall ist – muss diese korrekt als IP angegeben werden, also in der Form:

\begin{quote}
    \texttt{IP:192.168.40.144}
\end{quote}

und nicht als URI, also \texttt{URI:192.168.40.144}.

\begin{figure}[H]
    \centering
    \begin{subfigure}[b]{0.45\linewidth}
        \includegraphics[width=\linewidth]{14-TLS-Zertifikat-Extension-Tab.png}
        \caption{TLS Certificate – \emph{Extensions}-Tab}
    \end{subfigure}
    \hfill
    \begin{subfigure}[b]{0.45\linewidth}
        \includegraphics[width=\linewidth]{15-TLS-Zertifikat-SAT-IP.png}
        \caption{Subject Alternative Name (IP-Adresse)}
    \end{subfigure}
    \caption{TLS Certificate – Einstellungen im \emph{Extensions}-Tab}
    \label{fig:screenshot1415}
\end{figure}
