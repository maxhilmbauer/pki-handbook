\section{Import der Root-CA und des Intermediate-Zertifikats}
Damit der Client dem Server-Zertifikat vertraut, muss die Root-Zertifizierungsstelle (Root CA) lokal importiert werden.

\begin{enumerate}
    \item Öffnen der Anwendung \textbf{Keychain Access} (Schlüsselbundverwaltung).
    \item Wechsel in den Schlüsselbund \textbf{System}.
    \item Import der Datei \texttt{HaMa\_Root\_CA.crt}.
\end{enumerate}

\begin{quote}
    \texttt{File → Import Items → Root CA \& Intermediate CA auswählen}
\end{quote}

\begin{figure}[H]
    \centering
    \includegraphics[width=0.8\linewidth]{20-import-into-keychain.png}
    \caption{Import Root CA into Keychain Access}
    \label{fig:screenshot20}
\end{figure}

Anschließend muss das Zertifikat als vertrauenswürdig markiert werden.

\begin{figure}[H]
    \centering
    \includegraphics[width=0.8\linewidth]{21-ca-trust.png}
    \caption{Set Trust Settings for Root CA}
    \label{fig:screenshot20}
\end{figure}

\begin{tcolorbox}[colback=black!3!white, colframe=black!60!white, title=Hinweis]
    Der Import des Intermediate-Zertifikats ist optional, da es beim Verbindungsaufbau vom Server mitgeliefert wird.
    Dennoch empfiehlt sich der Import, um Zertifikatswarnungen im Browser zu vermeiden.
\end{tcolorbox}

\begin{figure}[H]
    \centering
    \includegraphics[width=0.8\linewidth]{22-intermediate-set-trust.png}
    \caption{Set Trust Settings for Intermediate Zertificate}
    \label{fig:screenshot20}
\end{figure}