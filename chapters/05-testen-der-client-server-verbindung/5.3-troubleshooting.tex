\section{Troubleshooting}
Im Verlauf des Tests traten mehrere Fehler auf, die im Folgenden analysiert und behoben wurden.

\subsection{TLS Handshake Failure}
\textbf{Fehler:}
\begin{verbatim}
sslv3 alert handshake failure
\end{verbatim}

\textbf{Ursache:}
Nginx unterstützte standardmäßig nur eine begrenzte Auswahl an Protokollen und Cipher-Suites.
Der verwendete Schlüsseltyp (ED25519) war nicht mit allen Clients kompatibel.

\textbf{Lösung:}
In der Nginx-Konfiguration wurden die unterstützten Protokolle angepasst:
\begin{verbatim}
ssl_protocols TLSv1.2 TLSv1.3;
ssl_ciphers   HIGH:!aNULL:!MD5;
\end{verbatim}

Nach einem Neustart von Nginx konnte der Handshake erfolgreich abgeschlossen werden.

\subsection{Ungültige Zertifikatskette}
\textbf{Fehler:}
\begin{verbatim}
unable to verify the first certificate
\end{verbatim}

\textbf{Ursache:}
Das Intermediate-Zertifikat wurde auf dem Server nicht korrekt verknüpft.

\textbf{Lösung:}
Server- und Intermediate-Zertifikat wurden zu einer vollständigen Kette kombiniert:
\begin{verbatim}
sudo cat server.crt HaMa_Intermediate_CA.crt \
    > /etc/nginx/ssl/server_fullchain.crt
\end{verbatim}
Anschließend wurde in der Nginx-Konfiguration das Fullchain-Zertifikat referenziert:
\begin{verbatim}
ssl_certificate /etc/nginx/ssl/server_fullchain.crt;
\end{verbatim}

\subsection{Cipher-Fehler im Browser}
\textbf{Fehler:}
Safari und Chrome lehnten die Verbindung mit ED25519-Schlüsseln ab.

\textbf{Ursache:}
Nicht alle Browser unterstützen ED25519 für TLS-Zertifikate.

\textbf{Lösung:}
Es wurde ein neues Server-Zertifikat mit RSA-Schlüsseln erzeugt.
Anschließend funktionierte die Verbindung ohne Fehlermeldung.

\subsection{Falscher Subject Alternative Name (SAN)}
\textbf{Fehler:}
Der Browser zeigte „Zertifikat ungültig für diese Website“ an.

\textbf{Ursache:}
Beim Erstellen des Zertifikats wurde im Feld \texttt{subjectAltName} eine \texttt{URI} anstelle einer \texttt{IP}-Adresse eingetragen.

\textbf{Lösung:}
Das Zertifikat wurde neu generiert mit:
\begin{verbatim}
subjectAltName = IP:192.168.40.144
\end{verbatim}

\section{Ergebnis}
Nach Durchführung aller Korrekturen wurde die Verbindung erfolgreich hergestellt.
Der Browser erkannte das Zertifikat als vertrauenswürdig, die HTTPS-Verbindung wurde ohne Warnmeldungen aufgebaut, und der Datenverkehr war vollständig verschlüsselt.

\begin{tcolorbox}[colback=green!5!white, colframe=green!50!black, title=Ergebnis]
    Die Client-Server-Verbindung über HTTPS war erfolgreich.
    Das Zertifikat wurde als gültig und vertrauenswürdig erkannt.
\end{tcolorbox}
