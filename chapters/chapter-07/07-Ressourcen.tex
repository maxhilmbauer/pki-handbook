\chapter{Planung}


\section{Sub heading for Subsection 01}
\newcolumntype{Y}{>{\raggedright\arraybackslash}X} % linkslastige Spalten

\begin{table}[htbp]
    \centering
    \renewcommand{\arraystretch}{1.4} % Zeilenhöhe erhöhen für Lesbarkeit
    \begin{tabularx}{\textwidth}{l Y}
        \toprule
        \textbf{Kategorie}                        & \textbf{Beschreibung}                                                                           \\
        \midrule
        Mikrocontroller / Datenmodul              & Konkrete Hardware festgelegt                                                                    \\
        Sensoren                                  & Sensordaten lokal ausgelesen                                                                    \\
        Akku / Stromversorgung                    & Hardware fest verkabelt und verlötet                                                            \\
        Gehäuse                                   & On-Device Funktionen implementiert                                                              \\
        Servergerät                               & z.B. Mini-PC oder leistungsfähiger Raspberry Pi zur lokalen Datenverarbeitung und als Webserver \\
        Access Point / Router                     & Zur Herstellung eines WLAN-Netzwerks zur Kommunikation zwischen Gerät und Server                \\
        USV (unterbrechungsfreie Stromversorgung) & Für den stabilen Betrieb des Servers und Netzwerks bei Stromausfällen                           \\
        \bottomrule
    \end{tabularx}
    \caption{Table Description}
\end{table}